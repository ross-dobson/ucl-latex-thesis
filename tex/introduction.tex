\documentclass[../main]{subfiles}
\begin{document}

\chapter{Introduction}

Welcome to the template. You'll notice this document has the \texttt{report} class. This means it can be laid out into parts, chapters, sections, subsections, subsubsections, and paragraphs. \textbf{This is designed for PhD theses, books, and lecture notes. As such, there is probably lots of shit you won't ever need to use in every scenario.} If you are just writing a short article, you may want to use \texttt{article} instead, which doesn't have parts and chapters. Luckily, I also have a template for that style too, please just ask.

I would recommend that rather than editing this template, you make a COPY of it first, and then edit that. That way, you can always come back to this which is known to work\footnote{If you have found something broken, please do let me know.} to check stuff.

I've tried to lay it out so you can quickly scan the contents to find whatever you need. Suitably, I've also tried to include an example of everything I've ever had to do/Google during my past work, which I hope encompasses everything you want to do.\footnote{Except Tikz for making your own diagrams, and any packages for Feynman diagrams, because I really can't be arsed to learn.}

In this document, I have used ``subfiles'', which allows you to use multiple .tex files rather than one HUGE \texttt{main.tex} file, which is particularly great at keeping your code neater, especially if you have a ridiculously large table to include---it can be hidden away in a seperate \texttt{.tex} file!

A very useful tool on Overleaf is the two blue arrow buttons between this code panel, and the PDF output panel. If you highlight code in the editor and click the right arrow, it takes you to that point in the PDF. If you select text in the PDF, and click the left arrow, it takes you to the code responsible, even if it's in a subfile. \emph{Very Useful}.

Not everything is useful to all situations. For example, I have included a \verb|\part|, which is the layout level above a chapter, even though you wouldn't necessarily use that even in a PhD thesis, probably only a book or lecture notes. But hey, I tried to make this document as useful as possible.

\end{document}