\documentclass[../main]{subfiles}
\begin{document}

\section{Equations}\label{sec:equations}

Basic stuff. Superscripts are done with the hat symbol \verb|^|, subscripts are done with the underscore \verb|_|. Don't forget to liberally wrap stuff in curly brackets. Here's what happens if you forget: $ t_{start}$ vs $t_start $. If you want to fix text in maths mode looking funky, wrap it in \verb|text{}|. E.g. $t_{end}$ vs $t_\text{end}$.


\subsection{Un-numbered equations}
If you just want to do a quick equation like $E=mc^2$, just wrap it in dollar signs in the text, e.g. \verb|$E=mc^2$|. If you want the equation to not be in-line in the text, you need to use the \verb|equation| environment. I've included one below, that also contains a \verb|split| environment showing that you can align equations by using \verb|&| symbols.

\begin{equation*}
    \begin{split}
        y &= mx+c\\
        E &= mc^{2}\\
        e^{i\theta} &= \cos{\theta} + i\sin{\theta}
    \end{split}
\end{equation*}

Notice this was an un-numbered equation, which was achieved by using the \verb|{equation*}| environment with an asterisk\footnote{Using an asterisk to remove numbers also works for chapters, tables, figures etc.} (you can achieve the same thing by wrapping maths in two dollar signs, i.e. \verb|$$y=mx+c$$|, but using the \texttt{equation} environment gives better control).  

\subsection{Numbered equations}
Having numbered equations also allows you to reference equations throughout your text:

\begin{equation} \label{eq:eulers_identity}
    e^{i\pi} + 1 = 0
\end{equation}

\subsection{Subequations}
Oh, subequations are also a thing:

\begin{subequations}
\label{eq:Maxwell}
Maxwell's equations:
\begin{align}
        B'&=-\nabla \times E,         \label{eq:MaxB} \\
        E'&=\nabla \times B - 4\pi j, \label{eq:MaxE}
\end{align}
\end{subequations}

\subsection{Referencing equations}
Because we included labels, I can reference them, e.g. look at equation \eqref{eq:eulers_identity}.
You can reference subequations too, e.g. \cref{eq:MaxB} and \eqref{eq:MaxE}. Notice with these two subequations, the first was referenced using cleverref with \verb|\cref|, the second was using \verb|\eqref|. The latter doesn't put `eq.' or `equation' before the number, which may be useful in different situations.

\section{Symbols}

\subsection{Vectors}

\subsubsection{Note on arrow vs. bold style vectors}
I hate vectors with arrows, so I disabled it in the mystyle.sty document, making them boldface instead. If you want the arrows back, simply comment out lines 134-137\footnote{I sometimes forget to update this when I add stuff. It should be \textit{somewhere} near those line numbers!} in \texttt{mystyle.sty}.

\subsubsection{Normal vectors}
You can do a vector with the \verb|\vec| command: $\vec{x}, \vec{y}, \vec{z}$. 

\subsubsection{Unit vectors} 
You can do unit vectors with the \verb|\hat| command: $\hat{x}, \hat{y}, \hat{z}$.

\subsection{Bars/average symbol}

This is simply \verb|\bar|. E.g. $\bar{x}$.

\subsection{Greek letter variants}
Here we're showing off all the variations on greek letters, putting the original next to the variant. $\epsilon \varepsilon$,  $\phi \varphi$, $\theta \vartheta$, $\rho \varrho$, $\sigma \varsigma$, $\pi \varpi$.

\subsection{Integrals \& Summations}
Here are the different ways of doing sums and integrals. (Note that even though we're not aligning the equations, they still need to be in a \verb|\split| environment for the line breaks to work):

\begin{equation*}
    \begin{split}
        \sum_{i=1}^{10} t_i \\
        \displaystyle\sum_{i=1}^{10} t_i \\
        \int_0^\infty \mathrm{e}^{-x}\,\mathrm{d}x\\
    \end{split}
\end{equation*}

Note the straight $\mathrm{d}$ on the $\mathrm{d}x$ achieved with \verb|\mathrm|, and the negative space between the integrand and $\mathrm{d}x$ given by \verb|\,|, which makes it look better.

\subsection{Brackets}
Here, we showcase the different types of brackets.
\begin{equation*}
    ( a ), [ b ], \{ c \}, | d |, \| e \|,
\langle f \rangle, \lfloor g \rfloor,
\lceil h \rceil, \ulcorner i \urcorner,
/ j \backslash
\end{equation*}

Here's an example of sizing brackets. Note how this:
\begin{equation*}
    \frac{\mathrm d}{\mathrm d x} \big( k g(x) \big)
\end{equation*}
is more readable than this:
\begin{equation*}
    \frac{\mathrm d}{\mathrm d x} \left( k g(x) \right)
\end{equation*} simply because we play with the height of the brackets slightly.

\subsection{Degree symbol}

$10^\circ{}$ versus $10\degree$. The first is bodged, the second uses the \verb|\degree| command added via \verb|gensymb|. It's better. Use it.

\section{Matrix}
Here we have a non-aligned \verb|{matrix}| and an aligned \verb|{matrix*}|:

\begin{equation*}
\begin{matrix}
  -1 & 3 \\
  2 & -4
\end{matrix}
 =
\begin{matrix*}[r]
  -1 & 3 \\
  2 & -4
\end{matrix*}
\end{equation*}

We can have bracket matrices with \verb|{pmatrix}| and determinant-style straight-edge matrices with \verb|{vmatrix}|.

\begin{equation*}
A_{m,n} = 
 \begin{pmatrix*}
  a_{1,1} & a_{1,2} & \cdots & a_{1,n} \\
  a_{2,1} & a_{2,2} & \cdots & a_{2,n} \\
  \vdots  & \vdots  & \ddots & \vdots  \\
  a_{m,1} & a_{m,2} & \cdots & a_{m,n} 
 \end{pmatrix*}
\end{equation*}

\begin{equation*}
A_{m,n} = 
 \begin{vmatrix*}
  a_{1,1} & a_{1,2} & \cdots & a_{1,n} \\
  a_{2,1} & a_{2,2} & \cdots & a_{2,n} \\
  \vdots  & \vdots  & \ddots & \vdots  \\
  a_{m,1} & a_{m,2} & \cdots & a_{m,n} 
 \end{vmatrix*}
\end{equation*}


A matrix in text must be set smaller:
$\bigl(\begin{smallmatrix}
a&b \\ c&d
\end{smallmatrix} \bigr)$
to not increase leading in a portion of text.

\end{document}
